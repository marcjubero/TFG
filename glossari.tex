\noindent
\textbf{API}: De l'anglès, Application Programming Interface. Conjunt de funcions i mètodes que es posen a disposició de l'usuari, com una capa d'abstracció,per a facilitar la interacció amb un determinat sofware o tecnologia.\\
\newline \textbf{REST}: De l'anglès, REpresentational State Transfer, model d'arquitectura enfocat a la comunicació, principalment emprat en el desenvolupament de \textit{web services} i com a alternativa de \textit{SOAP}. S'acostuma a emprar sobre \textit{HTTP}.\\
\newline \textbf{SOAP}: De l'anglès, Simple Object Access Protocol, és un protocol de missatgeria que permet a programes que s'executen en diferents màquines, comunicar-se entre si fent ús d'HTTP i XML.\\
\newline \textbf{HTTP}: De l'anglès, HyperText Transfer Protocol. Protocol de comunicació que permet la transferència d'informació en un sistema distribuït i col·laboratiu, com la World Wide Web (WWW), mitjançant un seguit de mètodes per sol·licitar i enviar informació, entre altres utilitats.\\
\newline \textbf{JSON}: De l'anglès, JavaScript Object Notation. Format lleuger d'intercanvi de dades basat en el llenguatge de programació JavaScript, fàcil d'entendre i escriure per humans i simple de generar i interpretar per les màquines.\\
\newline \textbf{Hash}: És un algorisme o funció per sumaritzar o identificar una dada a través de la probabilitat.\\
\newline \textbf{Blockchain}: Base de dades distribuïda que manté una llista ordenada de registres anomenats blocs en constant creixement.\\
\newline \textbf{Git}: Sistema de control de versions dissenyat per Linus Torvalds, pensat en l'eficiència i confiabilitat de manteniment de versions d'aplicacions amb una enorme quantitat de fitxers de codi font.\\
\newline \textbf{Repositori}: Lloc centralitzat on s'emmagatzema i manté informació digital.\\
