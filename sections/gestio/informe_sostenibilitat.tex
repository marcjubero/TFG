\section{Informe de sostenibilitat}
\label{gestio:informe_sostenibilitat}
Al llarg d'aquesta secció valorarem la sostenibilitat i el compromís social del projecte, analitzant-la des de tres punts de vista diferents: econòmic, social i ambiental.\\
Finalment otorgarem una puntuació de sostenibilitat sobre cada un dels aspectes anteriors des de tres punts de vista diferents: PPP\footnote{Projecte Posat en Producció}, vida útil i riscos.
\subsection{Sostenibilitat econòmica}
Com s'ha vist a la secció \ref{gestio:despeses}, gran part del gruix del pressupost pel projecte se l'emporta l'apartat de recursos humans. Això és degut a que s'han minimitzat al màxim els costos en tots els àmbits del projecte. Un exemple, és l'ús de \textit{blockchain} i serveis gratuïts en comptes de tercers de confiança.\\
\newline D'aquesta manera, amb la reutilització d'infraestructura, fent ús de llibreries i serveis ja existents, reduïm molt el temps de desenvolupament, fent que també es redueixi el cost global del projecte.
\subsection{Sostenibilitat social}
Com ja s'ha esmentat anteriorment, aquest projecte neix dins d'una empresa amb el clar l'objectiu de solucionar una problemàtica concreta dins d'aquesta. Per tant, a part de finalitats acadèmiques, té coma a finalitat donar una solució real.\\
\newline El desenvolupament del projecte millorarà el sistema d'emissió i signatura de consentiments informats de l'actual plataforma. De manera que els usuaris que facin servir la plataforma ja no s'hauran de desplaçar fins a la consulta del professional mèdic per autoritzar l'ús de les seves dades, guanyant, per part de l'usuari, una important quantitat de temps i qualitat de vida i, per part de l'empresa, una important millora en el sistema.
\subsection{Sostenibilitat ambiental}
L'ús de llibreries i serveis de tercers faciliten el desenvolupament i eviten destinar recursos addicionals per resoldre problemes concrets. Per tant, es minimitza també l'impacte ambiental del projecte.\\
\newline  A part, les màquines empleades per al desenvolupament del projecte, són màquines ja existents i utilitzades en altres projectes. L'ús de la virtualització permet la coexistència de diversos projectes en un mateix entorn, augmentant l'eficiència de les màquines on es troben allotjats.
\clearpage
\subsection{Matriu de sostenibilitat}
\begin{table}[h!]
  \centering
  \label{tab:matriu_sostenibilitat}
  \begin{tabular}{l c c c}
    	 & \textbf{PPP} & \textbf{Vida Útil} & \textbf{Riscos}\\
    	\midrule
    	\textbf{Ambiental}  & 9 & 9 & 0\\
    	\textbf{Econòmic}   & 7 & 7 & 0\\
    	\textbf{Social}     & 9 & 7 & -5 \\
    	\midrule
    	\textbf{Rang sostenibilitat} & 25 & 23 & -5\\
    	\bottomrule
    	\textbf{Total}      & & \textbf{43} & \\
  \end{tabular}
  \caption{Matriu de sostenibilitat}
\end{table}