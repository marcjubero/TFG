\section{Control de gestió}
Com ja s'ha mencionat repetides vegades al llarg del document, per al desenvolupament del projecte es fa ús de la metodologia àgil \textit{Scrum} conjuntament amb \textit{kanban}.\\
En el que a control de gestió es refereix, \textit{scrum} ofereix les eines necessàries per detectar problemes o desviacions.\\
\newline A mode de recordatori, \textit{Scrum} és una metodologia àgil que organitza les tasques a realitzar dins d'un projecte en el que s'anomenen \textit{sprints} o iteracions, i que aquestes es van succeint una després de l'altra fins acabar el projecte. Addicionlment, \textit{sprint} proposa la celebració de reunions diàries, per veure l'avanç del projecte, i reunions a l'inici de cada \textit{sprint} per tal de planificar-los i que tot l'equip sigui conscient de l'objectiu a seguir.\\
\newline En altres paraules, aquestes reunions inicials serveixen per determinar la feina que s'espera que es dugui a terme en el temps que duri l'\textit{sprint}.\\
\newline L'ús d'aquestes reunions diàries, permeten al \textit{project manager} tenir una visió de l'avanç i dels timings generals del projecte, podent d'aquesta forma detectar desviacions i actuar en conseqüència.\\
\newline Cal dir que no es considera desviació el fet de no completar en la seva totalitat les tasques especificades en un \textit{sprint}.\\
No obstant, la repetició d'aquest fet, si que es consideraria com a tal.\\
%Atès a les metodologíes àgils que es fan servir dins de l'empresa per al desenvolupament de projectes, en aquest cas \textit{Scrum}, es poden detectar desviacions dins de les planificacions de forma molt ràpida i eficaç i actuar en conseqüència en un temps relativament baix.\\
