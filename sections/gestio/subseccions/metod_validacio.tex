\subsection{Validació}
\label{metodologia:validacio}
%El mètode de validació que se segueix va lligat amb la metodologia \textit{Scrum}.\\
%\textit{Scrum}, defineix una serie d'iteracions a partir de les quals s'organitza la feina a realitzar al llarga de tot el projecte. En acabar cada una d'aquestes iteracions, per a que el codi sigui acceptat,  aquest ha de ser totalment funcional i presentar-se amb una bateria de tests que garantitzin el seu bon funcionament.
El mètode de validació que se segueix compleix dos criteris:
\begin{itemize}
    \item Compleix els requisits de la tasca
    \item La funcionalitat ha esta testejada adequadament.
\end{itemize}
Si es compleixen aquestes dos condicions, una de sentit comú, i l'altra donada per una de les metodologies de desenvolupament vistes anteriorment (\nameref{metodologia:tdd}), el codi es pot considerar apte.\\
El fet de tenir amb un codi testejat, assegura que si es fan canvis al codi es poden detectar canvis de comportament i errors fàcilment.

