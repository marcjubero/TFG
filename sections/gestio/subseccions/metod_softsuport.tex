\subsection{Programari de suport}
\label{metodologia:software_suport}
Per a facilitar la gestió del projecte es disposa del programari de la suite \textit{Atlassian}\footnote{https://www.atlassian.com/}. En concret es fan servir:
\begin{itemize}
    \item \textit{Jira}, per a la gestió de tasques. Disposa de \textit{plugins} per \textit{kanban} i \textit{scrum}. A part, permet obtenir gràfics i lectures sobre l'activitat al repositori, tant individual com d'equip.
    \item \textit{Bitbucket}, com a repositori de git. Permet la creació de branques i realitzar operacions sobre els mateixos repositoris.
    \item \textit{Confluence}, l'espai on l'equip documenta qualsevol incidència o millores.
\end{itemize}
Addicionalment,també es fan servir clients d'escriptori de git per agilitzar les operacions del dia a dia amb el repositori.\\
\newline Per a la integració contínua es fa servir \textit{Jenkins}\footnote{https://jenkins.io/}. Un programari que permet desplegar els últims canvis del sistema, que hi hagi repositori de forma automàtica.
%Com a suport per a la gestió d'\textit{Scrum}\cite{scrum} i del control de versions, així com de documentació interna, es disposa de llicència de la suite d'\textit{Atlassian}\cite{atlassian}, que ofereix diferents aplicatius:
%\begin{itemize}
%	\item \textbf{Jira}: per a la gestió de tasques i planificació de les iteracions.
%	\item \textbf{Confluence}: per a la documentació interna del projecte.
%	\item \textbf{Bitbucket}: com a repositori de control de versions.
%\end{itemize}
%Per altra banda, també es fa servir \textit{Jenkins}\cite{jenkins} per a la integració contínua.