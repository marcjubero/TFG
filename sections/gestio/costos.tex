\section{Costos}
Al llarg del projecte es poden agrupar les diferent despeses en tres categories diferents:
\begin{itemize}
	\item Equip de desenvolupament
	\item Maquinari
	\item Infraestructura
\end{itemize}

\subsection{Equip de desenvolupament}
Com bé indica el nom, aquestes despeses corresponen al pagament dels salaris de les persones que participen al projecte.\\
Dins de l'equip de desenvolupament hi podem trobar dos rols clarament definits:
\begin{itemize}
	\item Desenvolupador, l'estudiant que porta a terme el desenvolupament del projecte, i que per tant, dedica tota la seva activitat dins de l'empresa a realitzar les tasques pertinents.
	\item \textit{Project manager}, responsable de gestionar el projecte al llarg de tot el seu desenvolupament; des de l'especificació inicial fins a la integració amb la plataforma, alhora que assessora al desenvolupador en aspectes claus del projecte. 
\end{itemize}

\subsection{Maquinari}
Corresponent a les eines de treball que es posen a disposició de l'estudiant per tal que pugui dur a terme la seva activitat dins de l'empresa. \\
En aquest cas es tracta d'un Apple Mac mini amb un valor de compra de 796.28 \euro

\subsection{Infraestructura}
Les despeses d'infraestructura fan referència a totes aquelles despeses que van relacionades de forma més indirecta amb el projecte, ja que també es poden aplicar a l'activitat diària dins de l'empresa.\\
\newline Dins d'aquesta categoria queden incloses despeses tals com la mensualitat dels nodes de computació sobre els quals se suporta el projecte (25\euro/mes), llum, consultes legals als advocats, llicències de software, etc. que ascendeixen aproximadament a uns \EUR{150} mensuals.


\clearpage
\subsection{Taula de costos}
A la següent taula es pot veure, a mode de resum, les despeses del projecte.
\begin{table}[h!]
  \centering
  \label{tab:costos}
  \begin{tabular}{l l l l l}
    	Concepte & & \euro/hora & Cost Mensual & Cost Total\\
    	\midrule
    	\textbf{Personal} &  &  & \\
    	& Desenvolupador& 8 & (*) &9968\euro\\
    	& Project Manager & & \EUR{200} & \EUR{1600} \\
    	\textbf{Maquinari} & & - & \EUR{22,12} & \EUR{176,96}\\
    	\textbf{Infraestructura} & & \\
    	& Nodes computació & - & \EUR{25} & \EUR{200} \\
    	& Despeses vàries & - & \EUR{150} & \EUR{1200} \\
    	\bottomrule
    	\textbf{Total} & & & & \textbf{\EUR{13.144,96}}
  \end{tabular}
  \caption{Taula de costos.}
\end{table}
\newline En relació a les dades presentades a la taula anterior, cal tenir en compte unes breus consideracions:
\begin{itemize}
	\item Al llarg del projecte s'ha realitzat una ampliació d'hores del contracte, amb el respecetiu augment de sou, però mantenint sempre el preu per hora marcat per la Universitat Politècnica de Catalunya: \EUR{8}/hora.
	\item El cost total del maquinari és de \EUR{796.28}, però amb una amortització de 3 anys, suposa una despesa mensual de \EUR{22.12}.
\end{itemize}
Finalment, en relació al gantt presentat al lliurable anterior, cal dir que les tasques representades en el gràfic són independents al presupost presentat a la taula, ja que es tracta d'un pressupost tancat abans fins i tot de determinar les tasques que es duran a terme i que, en qualsevol dels casos, es tracta de despeses heretades del projecte al qual aquest TFG s'integrarà en un futur.