\section{Abast}
\label{gestio:abast}
%El projecte compta amb un abast totalment definit; s'ha de desenvolupar el mòdul de firma de consentiments informats que compleixi amb els requisits legals i funcionals per a posteriorment integrar-lo a la plataforma.\\
%\newline S'espera que al finalitzar el període de pràctiques a l'empresa, el mòdul quedi perfectament integrat i testejat amb la plataforma, amb l'objectiu que sigui plenament funcional i que es pugui fer servir sense dificultats en les ocasions que així ho necessitin.\\
%\newline Com a possibles obstacles, queda la investigació sobre les diferents metodologies i tecnlogies per a dotar al procés de firma del consentiment informat de validesa legal i jurídica, així com satisfer les necessitats de demostrar la inmutabilitat del document.
Com s'ha dit anteriorment, aquest projecte es centra en el desenvolupament d'un sistema per a l'emissió, validació i signatura de documents.\\
Per dur a terme el desenvolupament del projecte, es faran servir eines i serveis de tercers.\\
\\Com a resultat d'aquest treball, l'aplicació desenvolupada s'integrarà amb la plataforma \textit{Made of Genes} per a solucionar la problemàtica existent amb els consentiments informats.\\
\\Per això s'han de resoldre una sèrie de tasques definides a la secció \nameref{gestio:tasques} d'aquesta memòria, amb Gener de 2017 com a \textit{deadline}, que és el període establert per l'empresa.\\
\\Tot i que la llista de possibles obstacles no és extensa, a continuació es detallen els que podrien sorgir al llarg del desenvolupament del projecte:
\begin{itemize}
    \item Haver de fer un canvi de tecnologia
    \item Canvis de requisits legals a nivell de signatura
\end{itemize}
Aquests obstacles suposarien una desviació en el temps planificat per a la implementació del projecte.
%Como se ha dicho anteriormente, este proyecto se centra en la puesta en marcha del repositorio de metadatos, mediante una serie de herramientas Open Source, y en la automatización de la ingesta de metadatos durante la migración de los datos al Data Lake.\\
%\\Como resultado de este trabajo obtendremos un repositorio de metadatos semántico con el que podremos interactuar mediante una API, que servirá para optimizar el acceso a los datos almacenados en un Data Lake. Este repositorio se actualizará automáticamente durante los procesos ETL encargados de migrar todos los datos del Data Warehouse, actual sistema de la empresa, al Data Lake.\\
%\\Finalmente todos estos metadatos, guardados en el repositorio de metadatos semántico, estarán listos para que procesos de explotación del sistema Big Data de soporte al usuario, hecho a medida para el cliente, puedan utilizarlos para optimizar los procesos y el acceso a los datos. De manera que el repositorio de metadatos tiene un papel muy importante en la automatización de estos procesos de extracción de datos.\\
%\\Además se pondrá en funcionamiento una herramienta de visualización, mediante la cual el usuario podrá interactuar con el sistema Big Data, resultado de todo el proyecto Big Data Analytics Lab, de forma totalmente transparente. \\
%\\Para ello se han de cumplir una serie de objetivos objetivos, definidos en la sección \nameref{f-problema} de esta memoria, con Diciembre de 2015 como \textit{deadline}, que es el periodo establecido para la finalización de Big Data Analytics Lab, el proyecto del que forma parte este trabajo.\\
%\\No se ha previsto una larga lista de problemas o obstáculos que puedan surgir durante el desarrollo, más allá de los problemas habituales que puedan surgir al intentar integrar diferentes herramientas. Pese a que esta lista no es muy larga, seguidamente se detallan estos posibles obstáculos que se podrían encontrar durante el desarrollo del proyecto.
%\begin{itemize}
%\item Complicaciones a la hora de modificar algunas de las herramientas, resultando más difícil la implementación de las modificaciones necesarias en el código.
%\item Mayor dificultad a la hora de integrar algunas herramientas, como Virtuoso junto a la API o la comunicación entre API y Quarry.
%\end{itemize}
%Estos obstáculos provocarían una desviación en el tiempo planificado para la implementación del proyecto, cuyos costes se estudian en el apartado \nameref{control-gest} de la sección \nameref{s-costes} de esta memoria.