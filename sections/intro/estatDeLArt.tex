\section{Estat de l'art}
El consentiment informat, tal i com s'indica en apartats anteriors, és un procediment, que, tret de comptades ocasions, és d'obligada presència en l'àmbit mèdic.\\
\newline El procediment actual consisteix en un professional sanitari que informa al pacient de tots els possibles riscos, alternatives al tractament o anàlisis, així com dels possibles beneficis o resultats finals, de forma presencial. Un cop acabada la sessió informativa el pacient rep un contracte on, amb la seva signatura manuscrita afirma, amb ple ús de les seves facultats i sempre de forma totalment voluntària, haver rebut la informació, haver-la comprès i estar-ne d'acord.\\
\newline Un altre mètode de donar validesa legal, és la signatura electrònica, descrita amb anterioritat. Amb el temps i l'avanç de la tecnologia han aparegut empreses que busquen oferir serveis de certificació i firma electrònica tant a ususaris com a empreses.\\
\newline Empreses com \textit{Lleida.net\footnote{https://www.lleida.net/}} o \textit{Logalty\footnote{https://www.logalty.com/en/}} operen dins d'Espanya oferint serveis de certificació electrònica a través de la seva plataforma particular o a través d'una API que ells mateixos ofereixen.\\
\newline El consum dels serveis ofertats per aquestes empreses, les posiciona dins del rol de tercer de confiança, una entitat que actua com un notari online i que, mitjançant un certificat digital i un segell de temps certifiquen que un document ha estat emés en un moment i amb un contingut determinats. 
\newline Aquest procés, reconegut davant la llei, certifica la integritat del document, així com n'assegura el no repudi.\\
\newline L'ús de dispositius que capturin traç i pressió també està reconegut per la llei. El principal inconvenient d'aquests dispositius és, deixant de banda la pèrdua de la capacitat d'operar telemàticament, que el seu preu és molt alt, i la seva amortització resulta complicada.\\
\newline Finalment, a Espanya es disposa de sistemes de certificació com per exemple, el DNI electrònic, que ofereix als usuaris un certificat digital vàl·lid per a autenticar-se i per a signar electrònicament.
\newline Alternativament, des de ja fa un temps com a complement a l'esmentat e-DNI, existeix \textit{Cl@ve}, un sistema que busca facilitar la identificació dels usuaris davant de l'Administració, alhora que permet signatura electrònica mitjançant certificats.\\
\newline Els mètodes d'autenticació permesos al sistema \textit{Cl@ve} són mitjançant certificats (e-DNI) o bé mitjançant el que anomenen \textit{Cl@ve PIN}.\\
\newline Aquest segon mètode, és el que s'anomena contrassenya única o en anglès, \textit{One Time Password} (d'ara en endevant \textbf{OTP}). L'ús d'aquest mètode es basa en una contrassenya generada a partir de l'instant de temps en el que es sol·licita i, generalment, una clau privada i única de l'usuari, assegurant que per cada usuari i instant de temps, la contrassenya és única, garantint així, la identitat.\\
\newline Finalment i a mode de conclusió per aquest apartat, vistes les tecnologies anteriors, i entès el problema que es planteja donada la tipologia del negoci de l'empresa, s'ha arribat a la conclusió que existeix la necessitat d'implementar una solució pròpia que satisfagui els requisits particulars de la plataforma en la que en un futur proper s'integrarà al projecte.\\
\newline Aquesta conclusió ve donada per les "mancances" de les empreses que ofereixen serveis de certificació electrònica presentades anteriorment.\\
Aquestes mancances resideixen en una necessitat imperiosa de fer passar el procés de certificació per una pàgina web pròpia (de l'empresa certificadora) en la que permeten l'accés al document certificat, provocant d'aquesta forma un canvi de context dins de la plataforma \textit{Made of Genes} que des de l'empresa es considera inacceptable per a l'experiència d'usuari.