\subsection{Genètica i genòmica}
\label{estatArt:genomica_genetica}
Un gen és l'unitat bàsica d'informació biològica que s'implementa físicament sobre una molècula amb unes propietats químiques concretes, que reben el nom d'àcids desoxiribonucleics, altrament coneguts com ADN.\\
\newline L'ADN està format per una cadena nucleòtids que es diferencien els uns dels altres per la seva base nitrogenada, que se solen representar amb quatre lletres:
\begin{itemize}
    \item adenina (A)
    \item citosina (C)
    \item guanina (G)
    \item timina (T)
\end{itemize}
La seqüència d'aquestes quatre lletres (ACGT) serà el que posteriorment es coneixerà com a \textit{genoma}.\\
%\newline Quan parlem de la seqüenciació del \textit{genoma} d'un organisme, ens referim a la seqüència de les bases que trobem en una de les dues cadenes de nucleòtids que formen l'espiral de l'ADN. Només se'n pren una ja que la segona sempre és complementària a la primera.\\
\newline El color dels ulls i de la pell, diferents intoleràncies alimentàries o les propensions a determinades malalties, són característiques que diferencien a un organisme d'un altre i vénen donades per les diferents combinacions de nucleòtids.\\
\newline El conjunt de trets diferencials que mostra un organisme a nivell genètic, rep el nom de \textbf{genotip}, mentre que el conjunt dels seus gens, rep el nom de \textbf{\textit{genoma}}.\\
\newline El \textit{genoma} entre individus de la mateixa espècie pot variar entre ells un 0.1\%.\\
\newline Com a éssers vius, un dels processos genètics que ens caracteritzen, és la rèplica de l'ADN. Aquest procés es realitza milions i milions de vegades al llarg de la vida i es podria assumir com a perfecte.\\
\newline No obstant, no és infal·lible i dóna lloc a mutacions dins de la cadena de l'ADN.\\
\newline Quan ens trobem amb un gen que presenta mutacions a nivell de la seqüència parlem de diferents al·lels.\\
\newline En aquest cas ens trobem amb un gen amb la capacitat de fer la mateixa funció, però amb lleugers canvis. Per exemple, un canvi en la seqüència d’una proteïna pot provocar canvis en la seva eficiència, fent que realitzi la seva funció d'una forma més ràpida o més lenta.\\
\newline Hem anomenat genotip al conjunt de gens present en un individu donat, però com hem vist un mateix gen pot originar trets diferents en funció de diferents factors. Aquesta expressió diferencial del tret l’anomenem \textit{fenotip}, que es pot definir com l’expressió del genotip modulada per la interacció amb el medi.\\
\newline  Així, les dades del nostre genoma parlen sobre la nostra salut present i futura, sobre trets que heredem dels nostres pares i passem als nostres fills i, en general, una informació que pot ser utilitzada per finalitats potencialment discriminatòries. Per això, des de \textit{Made of Genes} es vol donar solució a les problemàtiques inherents a la genòmica, tant a nivell de privacitat com de la tecnologia necessària per la seve anàlisi.\\
\newline Com a curiositat, després d'aquesta breu introducció a la genòmica, cal esmentar que un genoma humà complet pot arribar a ocupar fins a 600GB de dades i que la seva anàlisi pot suposar més de 240 hores càlcul de CPU.\\
%\newline Després d'aquesta breu introducció a la genòmica, podem concloure que les dades que s'extreuen de la seqüenciació del \textit{genoma}, unes dades que parlen directament de qui, que i com és un individu, poden ser considerades extremadament sensibles, i que en determinades circumstàncies podrien ser usades amb fins poc lícits, tals com la discriminació, etc., per això, és necessari un model que garanteixi la privacitat de les dades emmagatzemades. Per satisfer aquesta necessitat, es fa ús d'un sistema de doble encriptació: per una banda, s'encripten les dades amb una clau privada de la pròpia empresa, única per a totes les dades desades, i per l'altre, s'encripten amb un clau única per cada client, que només aquest coneix. D'aquesta forma, s'assegura que en qualsevol cas, si les dades fossin obtingudes sense el permís de l'usuari, no podrien ser des-encriptades de forma convencional.\\
%\newline Per altre banda, a part de l'encriptació, una segona mesura per tal de preservar i garantir la privacitat de les dades és la dividir-les. D'aquesta forma, en cas d'una obtenció no autoritzada de les dades, el que sobté és un segment d'aquestes sense cap mena de relació amb l'anterior bloc de dades, resultant doncs, un bloc poc informatiu per si sol.\\
%\newline Després d'aquesta breu introducció a la genòmica, cal tenir en compte que  unes dades que per si soles parlen de qui, què i com és un individu, venen acompanyades de forma inherent per un seguit de problemàtiques a les que cal buscar sol·lució si el que es busca, és donar sortida a un camp amb un potencial tant gran.\\
%\newline Una primera problemàtica, és que unes dades tant sensibles com les que ens ocupen, podrien ser emprades per exemple, amb intencions discriminatòries, ja no parlem de color de pell ni color dels ulls ni de si un individu és més alt o més baix, si no del cas en que el la informació que ens dóna el seqüenciació, pot revelar certes propensions a una malaltia concreta i això ser usat com a pretext discriminatori cap aquella persona.\\
\newline Lligada a l'obtenció de tot aquest cúmul de dades, però aquests cop lligat als professionals sanitaris, és la possibilitat de descobrir de forma accidental una afecció que poc tingui a veure amb l'estudi o tractament que s'estigui duent a terme. En aquests casos, el professional sanitari està obligat a informar al pacient d'aquesta troballa i en moltes ocasions, pot ser un cas que estigui fora de les competències mèdiques del professional. Aquest fet, aquesta simple por a descobrir quelcom que compliqui el desenvolupament del tractament o estudi, impedeix l'avanç d'una tecnologia o forma de fer les coses que podria beneficiar a tothom.\\
\newline Per aquests motius, l'aposta de \textit{Made of Genes} és la de dividir les dades; tant en la forma de desar-les, com en la forma en la que es donen als especialistes que realitzen les anàlisis.\\
%En el primer dels casos, en cas de que hi hagi una vulneració de les dades i es pogués accedir a les dades emmagatzemades per part de l'empresa, s'accediria a blocs de dades que per si sols poden no tenir cap mena de valor científic i en as de tenir-lo, no revelar la informació sencera, ja que s'hauria de menester d'altres blocs de dades per acabar d'extreure la informació.\\
\newline En el primer dels casos, en el supòsit d'existir una vulneració de les dades emmagatzemades, possiblement es tindria accés a un bloc de dades que de forma individual pot tenir poc valor científic i, en cas de tenir-lo, revelaria poca informació o informació incompleta.
\newline Finalment, pel que respecta a com se serveixen les dades als professionals que realitzen les anàlisis, cada servei té unes dades genòmiques associades necessàries per efectuar l'estudi. Un cop el client autoritza l'ús d'aquestes dades, només es donen al professional
les indicades a l'especificació de l'estudi. Només es serviran les dades genòmiques al complet en el cas que un servei adquirit per un usuari les requereixi.
%\newline Finalment amb la genòmica apareix una problemàtica relacionada amb la informació que és capaç d'aportar.\\
%Es dona el cas de que durant els anàlisis es poden trobar anomalies o detectar possibles problemes en el pacient; i el professional sanitari encarregat de l'anàlisi es veurà obligat a informar del problema sense que aquest sigui molt possiblement del cap d'estudi inicial. Aquest fet, genera cert rebuig cap a un camp com és la medicina personalitzada.\\
%Per a donar solució a la problemàtica, es facilita única i exclusivament la informació necessària per a cada anàlisis, acotant així el possible descobriment de problemes no relacionats amb l'estudi i/o tractament. 

%\newline Així doncs podem dit que en el nostre \textit{genoma} hi ha escrit, per dir-ho d'alguna forma, qui som, les nostres característiques i allò que ens determina, biològicament parlant.\\
%Per un professional sanitari, la informació donaa per la seqüenciació de tot un \textit{genoma}, sovint és innecesària, ja que és més del que bonament necessita per als diagnòstics que desitja realitzar, i al accedir-hi corre el perill de descobrir possibles mutacions que derivin a malalties que no són competència seva però està obligat a manifestr al pacient a qui pertanyen les dades.\\