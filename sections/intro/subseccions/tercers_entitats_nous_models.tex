\subsection{Tercers de confiança, entitats de certificació i nous models}
\label{estatArt:thirdparty}
%\textbf{Oscar}: \textit{Tercers de confiança, entitats de certificació i nou models nou punt, aqui parles del canvi que hi ha des de les entitats de certificació actuals (i expliques els models centralitzats) i com mica en mica es van passant a models distribuïts, on el model per excelencia es:}
Partint de la idea de que el concepte criptografia existeix des de temps molt antics, molt abans de la informàtica, i amb la criptografia, models com el de clau pública-privada (\textit{PKI}) gairebé tan antics com la mateixa criptografia, no es d'estranyar que els principals models de certificació, validació i  signatura estiguin basats en tecnologies amb les que fa anys que convivim. Aquest és el cas de les anomenades "Autoritats de Certificació", entitats reconegudes i de confiança amb la capacitat d'emetre certificats reconeguts i amb validesa legal suficient com per ser irrefutables.\\
\newline Empreses com \textit{Lleida.net}\footnote{http://www.lleida.net/es} o \textit{Logalty}\footnote{https://logalty.com/} decideixen apostar per un model de negoci basat en oferir un servei que permeti la simplificació de processos com la certificació, validació i signatura de documents, fent que tot el procés d'obtenció de certificats i demés sigui transparent a l'usuari.\\
\newline Amb l'aparició d'internet i els sistemes distribuïts, el model \textit{PKI}, un model centralitzat i altament robust i consolidat, en el que es basa tota aquesta infraestructura, ha hagut d'adaptar-se als nous temps, fins i tot ha començat a veure's substituït en ocasions per models que permeten una major flexibilitat o una adaptació major a les necessitats d'aquests nous temps.\\
\newline Dins del marc d'aquest canvi, apareixen tecnologies com \textit{blockchain}, en el que es basa el desenvolupament d'aquest projecte de final de grau.