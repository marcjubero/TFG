\section{Context}
\label{context}
Aquest treball de final de grau s'ha realitzat sota la modalitat B és a dir, com a pràctiques a l'empresa i, per tant, analitzarem el context prenent com a referència l'activitat principal l'activitat de l'empresa.\\
\newline El perquè del projecte neix de la necessitat per part de l'empresa de resoldre quelcom dins del seu model de negoci, no obstant abans cal veure quins són els elements i/o entitats que intervenen en el projecte.

\subsection{L'empresa}
\textit{Made of Genes} és una empresa que ofereix un servei de genòmica personalitzada que posa a l'abast dels usuaris la seqüenciació del seu genoma i guardar-ne la informació de forma segura i de per vida.\\
\newline Per altra banda, l'empresa ofereix una plataforma online que actua com a \textit{marketplace} on es poden comprar aplicacions de terceres parts basades en el genoma.
Aquestes aplicacions estan disponibles perque aquelles persones que hagin contractat el servei de seqüenciació puguin treure partit de les dades enmagatzemades.\\
\newline La compra d'aquestes aplicacions/serveis, però, implica que les dades genòmiques dels usuaris són cedides a  tercers, i que aquests, amb les dades respondran als serveis contractats pels usuaris.\\
\newline Per assegurar que aquest procés sigui lícit, el pacient ha de ser conscient de què és el que està contractant i què implica la contractació de l'esmentat servei. Per això es fa ús del consentiment informat.
%\subsubsection{Rols a la plataforma}
%\begin{itemize}
%	\item \textbf{Pacient}: L'usuari final, aquella persona que compra el servei de seqüenciació juntament amb una o vàries aplicacions sobre les dades del genoma.
%	\item \textbf{Professional sanitari}: El professional que facilitarà la informació, tant la relativa al consentiment informat com la dels resultats del servei adquirit, a l'usuari final. Farà d'intermediari entre l'analista i l'usuari.
%	\item \textbf{Analista}: Aquell professional sanitari que farà ús de les dades cedides per l'usuari, en realitzarà les anàlisis i presentarà al professional mèdic un informe dels resultats.
%\end{itemize}

\subsection{Consentiment informat}
En l'àmbit mèdic, rep el nom de \textit{\textbf{consentiment informat}} el procediment a través del qual es garantitza que un pacient expressa de forma voluntària la intenció de participar en una investigació o tractament, havent prèviament comprès la informació que se li ha facilitat sobre l'estudi o tractament a realitzar, així com els beneficis, possibles riscos i alternatives i els seus drets i deures.\\
\newline En ocasions, i en contextos poc rellevants com podria ser un examen físic, aquest consentiment es pot arribar a sobreentendre i no requerir la presència d'un document. No obstant, en procediments invasius, que impliquin cert nivell de risc o bé amb alternatives, el consentiment informat s'ha de presentar per escrit i ha de ser signat pel pacient.\\
\newline Aquest document, serveix per autoritzar a les organitzacions, metges o professionals sanitaris en general, a dur a terme les operacions necessàries amb la seguretat de que el pacient, o la persona sobre la qual recaigui l'efecte del tractament o investigació, n'és conscient.

\subsection{Director i ponent}

\subsection{Autor}
L'autor d'aquest treball de final de grau forma part de l'empresa \textit{Made of Genes} dins de la qual té un rol de desenvolupador full stack.
Dins del que fa l'entorn del projecte, tindrà el rols de desenvolupador i \textit{tester}.\\
\newline És una de les part més interessades en acabar el projecte a temps, ja que el seu objectiu és el d'entregar el document final dins del període estipulat per acabar el Grau en Enginyeria Informàtica que està cursant.\\
\newline Com a treballador de l'empresa, el seu objectiu és el mateix que s'ha descrit anteriorment, a més dels objectius concrets de l'empresa.



