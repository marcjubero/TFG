\section{Frontend}
Un cop entesos el model de negoci i les necessitats a satisfer dins de la plataforma \textit{Made of Genes}, cal començar el disseny del mòdul  per a la seva posterior integració. A partir de tota la informació subministrada i els diferents casos d'ús es van començar a realitzar uns primers esboços que definien tot el model d'interacció amb l'usuari i com presentar les dades.\\
\newline En les següents apartats d'aquesta secció veurem com evoluciona el frontend de l'aplicació en el transcurs del projecte.
\subsection{Primers passos}
Com bé s'ha descrit en capítols anteriors, la part del client de la plataforma es desenvolupa amb \textit{AngularJS}.\\
Inicialment, l'experiècnia de l'autor amb tecnologies d'ambit web era més aviat inexistent, així que mentre es recollien les dades necessàries per a començar amb els esbossos i el disseny del document de requisits finals, es va aprofitar per indagar i recopilar informació sobre el llenguatge (Javasctipt), sobre el qual no es disposava de l'experiència suficient per a dur a terme amb soltura un projecte d'aquest caire així com agafar una mica de pràctica amb lús i beneficis del \textit{framework} que es fa servir a l'empresa.
\subsection{Inici del desenvolupament}
Un cop redactat el document de requisits, especificats els diferents casos d'ús i tenir una primera versió dels diferents dissenys, comença el que es pot considerar una primera fase de desenvolupament.
\newline En aquesta fase, es porta a terme un projecte des de zero.
\newline El desemnvolupament d'aquest projecte suposa el gruix principal del dsencolupament del frontend, ocupant gran part del temps dedicat.
\subsection{Fase final}
Finalment, es porta aterme la esmentada integració del mòdul que forma part del TFG a la plataforma \textit{Made of Genes}.
\newline Aquest procés d'integració suposa tot un seguit de canvis dins del projecte inicial.
\newline Tot i els canvis, no hi han hagut deviacions que hagin afectat seriament al procés del projecte, i s'ha pogut finalitzar a temps

% explicar que es planteja com una finestra emergent que permet a l'usuari signar el consentiment informat que se li mostra mitjançant la seva contrassenya  de la plataforma i un codi OTP 