\section{Frontend}
\label{frontend}
Amb una cerca ràpida a \textit{Google} de la paraula \textit{frontend}, trobem la següent definició sota l'etiqueta \textit{computing}:
\begin{displayquote}
"\textit{\textbf{adj.} (of a device or program) directly accessed by the user and allowing access to further devices or programs.}"
\end{displayquote}
Tenint en compte l'anterior definició podem sobreentendre que la paraula \textit{frontend} fa referència a la part del projecte amb la que l'usuari interactuarà de forma directe, aquella que ha de permetre el visualitzar, signar i validar els consentiments informats que es generin amb la compra dels serveis a través de la plataforma; i fer-ho de la forma mes usable possible.\\
\newline El \textit{frontend} s'ha desenvolupat amb \textit{AngularJS}\footnote{https://angularjs.org/}, un dels frameworks Javascript més estesos en el panorama \textit{web-development} actual.

\subsection{AngularJS}
\textit{AngularJS} és un framework per al desenvolupament d'aplicacions web dinàmiques, caracteritzat per ser de codi obert, basat en javascript i sobretot, mantingut principalment per \textit{Google} i una àmplia comunitat que s'esforça en donar solució als reptes que sorgeixen en el desenvolupament de les anomenades \textit{SPA} (\textit{Single Page Application}).\\
\newline Un altre punt que ha posicionat fortament \textit{AngularJS} com un dels \textit{frameworks} per antonomàsia en el que fa a desenvolupament web, és l'us de l'arquitectura MVC (Model Vista Controlador), que ens permet distribuïr el codi i funcionalitats d'una forma coherent i fàcilment testejable.\\
\newline Cap a finals de 2016 es va alliberar una versió estable del que s'anomena \textit{Angular2}, una revisió integral del framework web. Aquesta revisió suposa una reescriptura total del mateix, així com el canvi de funcionament de molts mòduls del propi \textit{framework}. Aquest pot ser un dels principals motius que l'adopció d'aquesta segona versió sigui lenta i costosa. \\
\newline Actualment es consideren versions suportades les versions 2.4.1 i 1.6.1, sent aquesta última la més emprada sempre que es parla d'\textit{AngularJS}.\\
\newline A grans trets, \textit{AngularJS} extén l'HTML aportant tot un seguit d'etiquetes, anomenades directives, que permeten als desenvolupadors coses com:
\begin{itemize}
    \item Control d'estructures del DOM
    \item Amagar i mostrar elements del DOM
    \item Validació dinàmica de formularis
    \item Afegir nous comportaments a elements del DOM (gestió d'events...)
    \item Reutilitzar HTML  agrupant-lo en components
\end{itemize}

\subsubsection{Single Page Application}
El terme \textit{single page application} (SPA) es fa servir principalment amb aplicacions web que ofereixen un tipus d'interacció dinàmica i semblant al que trobariem en aplicacions mòbils o d'escriptori.\\
\newline La principal diferècia entre una pàgina web corrent i una \textit{SPA}, és que aquesta última refresca molt poques vegades la pantalla. Una \textit{SPA} fa un ús intensiu d'AJAX, una forma de comunicar-se amb la part back-end servidora sense necessitat de refrescar la pàgina, que permet carregar les dades directament dins de la nostra aplicació, fent que aquestes es renderitzin directament sobre client.\\
\newline En altres paraules, quan l'usuari accedeix per primera vegada a una \textit{SPA}, el navegador web descarregarà tota l'aplicació. Un cop descarregada, a mesura que l'usuari navegui, l'aplicació anirà descarregant la informació necessària i l'anirà pintant a mesura que faci falta i com faci falta. D'aquesta forma, s'aconsegueix una millor experiència d'usuari, ja que el que l'usuari final 

\subsubsection{MVC}
El patró anomenat model vista controlador (mvc) és un patró d'arquitectura del software que proposa el separar els components que formen el software en tres grans blocs. Per una banda el \textbf{model}, que correspon a tots els objectes de la lògica de negoci, també hi ha la \textbf{vista}, altrament dit interfície d'usuari o bé canal de comunicació amb altres programes, i finalment, el \textbf{controlador}, que gestiona tot el flux d'informació així com la comunicació entre el model i la vista.\\
\newline Portat al context que ens ocupa, \textit{AngularJS}, es tradueix en tot un seguit de vistes, fitxers purament HTML, gestionades per controladors, fitxers Javascript, que vinculen dades i en controlen el comportament, i una bateria de serveis i llibreries, també fitxers Javascript, que són les encarregades de realitzar la comunicació amb el "món exterior".


