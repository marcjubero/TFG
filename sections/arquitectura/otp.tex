\section{One Time Password}
La contrasenya d’un sol ús, altrament anomenada \textit{One Time Password}, en endavant OTP, és una contrasenya que tal i com indica el seu nom, és vàl·lida per una sola vegada. Un cop emprada, se n’ha de generar una de nova o bé demanar al proveïdor pertinent que en faciliti una de nova;en qualsevol dels casos un cop feta servir, se n’ha d’adquirir una de nova.\\
\newline Aquest tipus de pràctica apareix a partir de les cada cop més evidents mancances de la contrasenya estàtica original, que es veuen accentuades amb l’imparable creixement que d’internet.\\
\newline L’ús principal dels OTP es troba en el que s’anomena autenticació en 2 passos, un procés que busca assegurar la identitat de l’usuari mitjançant un segon pas en el procés convencional d’autenticació clàssic a partir de nom d’usuari i contrasenya, consistent en la introducció d’un codi que només l’usuari que s’autentica coneix.\\
\newline La implementació que se'n fa dins del projecte es basa en aquella especificada a l'RFC6238, que porta per títol \textit{TOTP: Time-Based One-Time Password Algorithm}.\\
\newline L'anterior document descriu una extensió de l'algorisme de OTP,descrit a l'RFC4226, amb títol \textit{HOTP: An HMAC-Based One-Time Password Algorithm}